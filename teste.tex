\documentclass[10pt,reqno]{amsart}

\usepackage{amsrefs}
\usepackage{amsmath}
\usepackage{amssymb}
\usepackage{hyperref}
\usepackage[utf8]{inputenc}
\usepackage{stackrel}
\usepackage{color}
\usepackage{graphics,graphicx}
\usepackage{comment}

\newtheorem{theorem}{Theorem}[section]
\newtheorem{lemma}[theorem]{Lemma}
\newtheorem{example}[theorem]{Example}
\newtheorem{examples}[theorem]{Examples}
\newtheorem{definition}[theorem]{Definition}
\newtheorem{conjecture}[theorem]{Conjecture}
\newtheorem{false}[theorem]{False result}
\newtheorem{proposition}[theorem]{Proposition}
\newtheorem{remark}[theorem]{Remark}
\newtheorem{corollary}[theorem]{Corollary}
\newtheorem{observation}[theorem]{Observation}
\newtheorem{nothing}[theorem]{ }

\DeclareMathOperator{\supp}{supp}
\DeclareMathOperator{\dist}{dist}
\DeclareMathOperator{\Hess}{Hess}
\DeclareMathOperator{\spann}{span}
\DeclareMathOperator{\id}{id}
\DeclareMathOperator{\epi}{epi}
\DeclareMathOperator{\tr}{tr}
\DeclareMathOperator{\dom}{dom}

\begin{document}
 
Teste

Modificação

\end{document}

%% A -
%% a - 
%% B - bola
%% b - 
%% C - Control set, Cartan tensor
%% C -
%% c -
%% \mathcal C 
%% d - usual derivative, subdifferential
%% d_M - métrica
% D - 
% D -
%% e - 
%% E - 
%% \mathcal E - extende geodesic field
%% f - function
%% g - Riemannian metric and fundamental tensor, elemento do grupo de Lie
%% \mathfrak g - álgebra de Lie
%% G - Lie group, 
%% \mathcal G - Geodesic spray
%% F - Finsler/Minkowski/assymetric norm
% \mathcal F
% h -
%% \mathfrak h - álgebra de Lie
% H - Hamiltonian, subgrupo
% J - 
%% k - índice
%% l - distinguished section, extremidade de intervalo;
%% M - manifold
%% N - manifold
%% n - dimensão
%% R - raio
%% S - esfera 
%% t - parâmetro temporal
% u - controle
%% U - horizontal open subset
%% V - espaço vetorial
% W 
%% x - horizontal direction
% X - 
%% y - vertical direction
% Y - 
%% z - vertical direction
% Z - 
% \alpha - 
% \beta - 
% \epsilon - interval extremal
% \varepsilon - parameter
%% - \eta - 
% \gamma - 
% \kappa - 
%% \Gamma - 
%$ \lambda - 
%% \phi - coordinate system
% \Phi - 
% \omega - 
% \delta - 
%% \tau - 
% \nabla
% \psi - 
% \sigma - 
%% \theta - 
% \varphi - 
%% \zeta
